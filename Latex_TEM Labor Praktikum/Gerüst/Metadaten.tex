% hier werden alle Gruppen- und Versuchspezifischen Daten eingetragen

% Namen der Studenten eintragen:
\newcommand{\studentnameeins}{Gruppe 5a}
\newcommand{\studentnamezwei}{Juan Bettinelli}



% Name des Betreuers eintragen:
\newcommand{\betreuer}{AG Lehmann und ZELMI}
\newcommand{\betreuerzwei}{}
% Thema des Versuchs
\newcommand{\versuch}{TEM Laborpraktikum} % hier das Thema eintragen
\newcommand{\untertitel}{}		     % hier Untertitel des Themas eintragen, wenn erforderlich
% Datum des Versuchs z.B.: 24. November 2019
\newcommand{\datumversuch}{4.12 2022 }
% Semester z. B. WS 2019/20; SoSe 2020
\newcommand{\semester}{WiSe 2021-2022}
\newcommand{\version}{}

% Weitere Befehle, die man gebrauchen könnte
\newcommand{\anf}[1]{\glqq #1\grqq{}} % Anführungszeichen um ein Wort
\newcommand{\bildunterschrift}[2]{\caption[#1]{\textbf{#1:} #2}} % Bildunterschrift mit eigener Überschrift
% gerne ergänzen...