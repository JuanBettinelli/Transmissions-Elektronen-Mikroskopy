\begin{thebibliography}{xx}


\bibitem[1]{1}
K. Arts, H. Thepass, M. A. Verheijen, R. L. Puurunen, W. M. M. Kessels, and H. C. M. Knoops, Chemistry of materials : a publication of the American Chemical Society 33, 13 (2021).
\bibitem[2]{2}
 M. L. Lehmann, EM 1&2: Elektronenmikroskopie 1 & 2: Vorlesungen und Presenationen (Berlin, 2021).
\bibitem[3]{3}
Dirk Berger ZELMI (2022).
\bibitem[4]{4}
Dirk Berger ZELMI (2022).
\bibitem[5]{5}
Tore Niermann and Juan Bettinelli (2022).
\bibitem[6]{6}
J. Bettinelli, Probenprepation Messdatenprotokoll TEM-Labor (2022).
\bibitem[7]{7}
J. Bettinelli, TEM proben Vorbereitung 1,2&3 Mitschrift (2022).
\bibitem[8]{8}
M. L. Lehmann, TEM-Laborpraktikum 2022 Aufgaben- und Zeitplan (2022).
\bibitem[9]{9}
Dirk Berger ZELMI, STEM-EDX Praktikum ZELMI 2022 07 25: Theorie STEM EDX (Berlin, 2022).
\bibitem[10]{10}
Frederik Otto, TEM column and TEM operation: TITAN Bedinung (Berlin, 2022).
\bibitem[11]{11}
Juan Bettinelli, Holographie Labor Mitschirft (2022), Holographie Labor Mitschirft. Accessed 1 November 2022.
\bibitem[11]{11}
Juan Bettinelli, TEM Beugung Labor Mitschrift (2022). Accessed 1 November 2022.
\bibitem[11]{11}
Juan Bettinelli, TEM Holographie Labor Mitschrift (2022). Accessed 1 November 2022.
\bibitem[12]{12}
S. Sören, Präparation von Proben für TEM-Untersuchungen: Präparation von Proben für TEM-Untersuchungen (Berlin, 2022).
\bibitem[13]{13}
T. Winkler, TEM Beugung Mitschrift 2 (2022). Accessed 16 November 2022.
\bibitem[14]{14}
Tolga, Holographie Tafel aufschrift (2022).
\bibitem[15]{15}
Juan Bettinelli and Till Winkle, STEM Labor Mitschrift Gruppe 5 Scan (2022).
\bibitem[16]{16}
TITAN and Frederik Otto, TITAN Ein/Ausscheck verfahren: TITAN Bedinung (Berlin, 2022). Accessed 1 November 2022.
\bibitem[17]{17}
Dirk Berger ZELMI, Theorie STEM-EDX 2022 06 20.pptx: Versuchsteil STEM-EDX (2022).
\bibitem[18]{18}
Frederik Otto, Beugung Tafel Foto (TU Berlin). Accessed 1 October 2022.




\end{thebibliography}